\chapter{Global Functions in SST Python Module}
\label{chap:globals}

The SST core python module provides a set of global functions not
attached to any particular class. These functions generally fall into
one of the following categories: general control and informational
functions, functions to get handles to existing objects and statistic
enable and control functions.  These functions are described below.
Following those descriptions is a section on general notes on
statistics (Section~\ref{sec:gen-notes-stats}).

\section{General Control and Informational Functions}
\label{sec:sim-control}

\begin{functiondoc}{setProgramOption(option,value)}{
    Sets the specified program option for the simulation.  These
    mirror the options available on the sst command line.  Parameters
    set in the python file will be overwritten by options set on the
    command line.  Use sst --help to get a list of available options.
}

  \param{option} (type: string) configuration option to set

  \param{value} (type: varies by option) value to set option to

  \noreturn
\end{functiondoc}


\begin{functiondoc}{getProgramOptions()}{
    Returns a dictionary with the current values of the program
    options.  This will include all program options, not just those
    set in the python file.  }

  \returns python dictionary with program options and values
\end{functiondoc}


\begin{functiondoc}{getMPIRankCount()}{
    Returns the number of physical MPI ranks in the simulation
}
  
\returns number of MPI ranks in the simulation
\end{functiondoc}


\begin{functiondoc}{getSSTThreadCount()}{
    Returns the threads per rank specified for the simulation
}

  \returns number of threads per MPI rank in the simulation
\end{functiondoc}


\begin{functiondoc}{setSSTThreadCount(threads)}{
    Sets the number of threads per rank for the simulation.  These
    values can be overwritten by using -n on the command line.
}

  \param{threads} (type: int) number of threads per MPI rank to use in the
  simulation

  \noreturn
\end{functiondoc}

\begin{functiondoc}{pushNamePrefix(prefix)}{
    Pushes a name prefix onto the name stack.  This prefix will be
    added on the names of all Components and Links.  The names in the
    stack are separated by a period.  Example, if
    pushNamePrefix(“base”) and pushNamePrefix(“next”) were called in
    that order, the prefixed name would be “base.next”.  Prefixes can
    be popped from the stack using popNamePrefix().
}

  \param{prefix}: (type: string) prefix to add to the name stack

  \noreturn
\end{functiondoc}


\begin{functiondoc}{popNamePrefix()}{
    Pops a prefix from the name stack.  See pushNamePrefix for how
    name stacks are used.
  }

  \noreturn
\end{functiondoc}


\begin{functiondoc}{exit()}{
    Causes the simulation to exit.
}

  \noreturn
\end{functiondoc}
  

\section{Functions to Get Handles to Existing Objects}
\label{sec:handles}

\begin{functiondoc}{findComponentByName(name)}{
    In many cases, Components and SubComponents will be created using
    library functions and the user will not have direct access to
    their handles.  In some instances, the provided python modules
    will have accessor functions that can provide handles to these
    elements.  If this is not provided by the library, the user can
    call the findComponentByName() function to get a handle to the
    desired element.  The function can find handles for both
    Components and SubComponents.  The use of this function
    presupposes a knowledge of the naming convention of the elements
    in the build functions of the library.
}

  \param{name} (type: string) name of the Component or SubComponent to
  find.  The name for SubComponents is described above.  Slot indexes
  are optional in cases where only on SubComponent has been added to a
  slot, but you can also use [0] in all cases, even when the actual
  name will not display this way.

  \returns the function will return a handle to the
  Component/SubComponent with the provided name, or None if the name
  is not found.
\end{functiondoc}


\section{Statistic Enable and Control Functions}
\label{sec:stat-enable}

The following functions are used to enable statistics on Components
and SubComponents using the name or type of the element.  See “General
Notes on Statistics” (Section \ref{sec:gen-notes-stats}) below for more
information.

\begin{functiondoc}{enableAllStatisticsForAllComponents(stat_params_dict)}{
    Enables all statistics for all Components in the simulation that
    have already been instanced.
}
  
  \param{stat_params_dict} (type: dict) Python dictionary that specified the
  statistic parameters.  All statistics will get the same set of
  parameters.

  \noreturn
\end{functiondoc}


\begin{functiondoc}{enableAllStatisticsForComponentName(name, stat_params_dict,
    apply_to_children=False)}{

    Enables all statistics for the Component named in the call.  This
    call works for both Components and SubComponents.
}

  \param{name} (type: string) name of the Component or SubComponent on
  which to enable all statistics.  The name for SubComponents is
  described above.  Slot indexes are optional in cases where only one
  SubComponent has been added to a slot, but you can also use [0] in
  all cases, even when the actual name will not display this way.  If
  component with the provided name not found, the function will call
  fatal().

  \param{stat_params_dict} (type: dict) Python dictionary that
  specified the statistic parameters.  All statistics will get the
  same set of parameters

  \param{include_children} (type: bool) If set to True, will
  recursively enable all statistics on all SubComponent descendants of
  named element.

  \noreturn
\end{functiondoc}


\section{General Notes on Statistics}
\label{sec:gen-notes-stats}

