\section{Link}

The Link object is used to connect Component/SubComponents together to
form the simulation.  The Link is created using:

\begin{functiondoc}{Link(name, latency=None)}{
    Creats a new Link object
}

  \param{name} (type: string) Name of the link

  \param{latency} (type: string) default latency for the link,
  optional.  This will be used if no latency is specified in calls to
  Link.connect() or (Sub)Component.addLink().

  \returns{Link object}

\end{functiondoc}


\begin{functiondoc}{connect( (comp1, port1 latency1=default), (comp2, port2, latency2=default) )}{

    Connects two ports using the link object.

    Actual parameters are two tuples representing the information for
    the ports to be connected.  The fields in the tuple are (comp,
    port, latency) as describe below.

}

  \param{comp} (type: Component or SubComponent)
  Component/SubComponent object that the port is a part of

  \param{port} (type: string) port to connect to

  \param{lantency} (type: string) latency of link from the perspective
  of the corresponding Component/SubComponent sending an event.  This
  is optional, and if not specified, the default latency of the link
  will be used.  If no latency is set, either in the call or as a
  default, the call will fatal.

  \noreturn

\end{functiondoc}


\begin{functiondoc}{setNoCut()}{
    Tell the simulator that this link should not be “cut” by a
    partition boundary.  In effect, it will guarantee that the two
    Components connected by this link will be on the same rank when
    using an autotmatic partitioning scheme (this attribute is ignored
    if the self partitioner is used).  This must be used with care, as
    you can easily get into a situation where too many components are
    on the same rank.  }

  \noreturn
\end{functiondoc}
