\documentclass[11pt,letterpaper]{report}

\usepackage[utf8]{inputenc}
\usepackage[english]{babel}
\usepackage[english]{isodate}
\usepackage[parfill]{parskip}
\usepackage{listings}
\usepackage{xcolor}
\usepackage{graphicx}
\usepackage{subfigure}
\usepackage{hanging}


\addtolength{\oddsidemargin}{-.75in}
\addtolength{\evensidemargin}{-.75in}
\addtolength{\textwidth}{1.5in}

\addtolength{\topmargin}{-.75in}
\addtolength{\textheight}{1.5in}


%% The following are functions/macros to make regions where
%% underscores don't have to be escaped with \.  One version,
%% \escapeunderscore{} will replace all underscores found with
%% \textunderscore.  The version \escapeunderscoremath{} will replace
%% all underscores, except those in a math region defined by $ $ with
%% \textunderscore.
\makeatletter
%\newcommand{\escapeunderscoremath}[1]{\expandafter\@find@math#1$\relax}

%% \def\@find@math#1$#2\relax{%
%%   \ifx \relax #2\relax
%%     % #2 is empty, so just call @escape@underscores.  Need to add _ to
%%     % #end so it can terminate if it doesn't find one.
%%     %\typeout{No symbol found}
%%     \@escape@underscore#1_\relax%
%%   \else
%%     % Output what we've seen already, including the $
%%     \@escape@underscore#1_\relax%
%%     $
%%     % Need to look for end $.  There's already a $ at end, so no need
%%     % to add one
%%     \@end@math#2\relax
%%   \fi
%% }

%% \def\@end@math#1$#2\relax{%
%%   \typeout{#1}
%%   \typeout{#2}
%%   \ifx \relax #2\relax
%%     % #2 is empty, this is an error condition
%%     #1%
%%   \else
%%     % Simply output text followed by $
%%     #1%
%%     $
%%      % Now, look for more math sections
%%     \@find@math#2\relax
%%   \fi
%% }      
    
% \expandafter for the case that the filename is given in a command
\newcommand{\escapeunderscores}[1]{\expandafter\@escape@underscore#1_\relax}

\def\@escape@underscore#1_#2\relax{%
    \ifx \relax #2\relax
        % #2 is empty => finish
        #1%
    \else
        % #2 is not empty => underscore was contained, needs to be replaced
        #1%
        \textunderscore
        % continue replacing
        % #2 ends with an extra underscore so I don't need to add another one
        \@escape@underscore#2\relax
    \fi
}
\makeatother
%% End of underscore macros


%% colors for code listings

% color of function signature in functiondoc environment
\definecolor{funcdef}{HTML}{0047AB}
% background color for example code listings
\definecolor{codeexbkg}{HTML}{F7EFDE}
%\definecolor{codeexbkg}{HTML}{F7E7CE}
%\definecolor{codeexbkg}{HTML}{708090}

% syntax colors:
\definecolor{keywordcolor}{HTML}{008000} % dark green
%\definecolor{commentcolor}{HTML}{D2B48C} % tan
%\definecolor{commentcolor}{HTML}{918151} % dark tan
\definecolor{commentcolor}{HTML}{88794C} % dark tan
%\definecolor{stringcolor}{HTML}{FFA500} % orange
\definecolor{stringcolor}{HTML}{FF8C00} % dark orange
%\definecolor{identifiercolor}{HTML}{CC6677}
\definecolor{identifiercolor}{HTML}{993C4D}
\definecolor{builtincolor}{HTML}{009ACD}

\lstset{
    basicstyle=\small\ttfamily,
    %backgroundcolor=\color{yellow},
    %numbers=left, numberstyle=\tiny, stepnumber=2, numbersep=5pt,
    keywordstyle=\color{keywordcolor}\bfseries,
    commentstyle=\color{commentcolor}\textit,
    stringstyle=\color{stringcolor},
    %identifierstyle=\color{identifiercolor},
    %showstringspaces=false,
    %keywordstyle=\color{blue}\bfseries,
    %morekeywords={align,begin},
    %pos=l
}

%% used to create each item of the lists.  The variable field uses
%% \escapeunderscore, what comes after does not
\newcommand{\param}[1]{\item[\textbf{\escapeunderscores{#1}}]} 

%% this is used to specify the return value, which must be the last
%% item of the parameter list because it will end the description
%% environment.  You must use \returns or \noreturn.
\newcommand{\returns}[1]{\item[\textbf{returns}] #1 \end{description}}

%% this is used to specify there is no return value, which must be the
%% last item of the parameter list because it will end the description
%% environment.  You must use \returns or \noreturn.
\newcommand{\noreturn}{\item[\textbf{returns}] no return value \end{description}} %%%% this is used to create each item of the lists
%%%% the monotype font is used to ensure consistent horizontal spacing


%\newenvironment{parameters} %%%% this adjusts spacing as I desire
%    %%{\begin{itemize}[leftmargin=*] \itemsep2pt \parskip0pt \parsep0pt}
%    {\begin{description}}
%    {\end{description}} 


%% Environment used to document a function.  This will keep the format
%% uniform.

%%   First argument: type signature of function being documented.
%%   This will be wrapped in \escapeunderscore{} to avoid having to
%%   use \_, since we use underscores regularly in variable names.

%%   Second argument: description of the function.  Since this may be
%%   multiple paragraphs, it is not wrapped in \escapeunderscore{}.
%%   You can do this manually for each paragraph if you don't want to
%%   use \_, or you can just use \_.
\newenvironment{functiondoc}[2]% environment name 
{% begin code 
  \par\vspace{\baselineskip}%\noindent

  {

    \begin{hangparas}{.25in}{1}
    %% \raggedright is because we often have long variable names and
    %% we want it to wrap early rather than overflow line
    \raggedright\large\bfseries\color{funcdef}\escapeunderscores{#1}
    \end{hangparas}
  }
  
  #2

  \textbf{\underline{Parameters}:} %

  \begin{description} %\leftmargin=1in
    \setlength{\listparindent}{\parindent}%
    \setlength{\itemindent}{\parindent}%
    \setlength{\parsep}{\parskip}%

  
  %\par\vspace{\baselineskip}\noindent\ignorespaces 
}% 
{% end code
  %\end{description}%\ignorespacesafterend
  %%\hrulefill
  \par\vspace{\baselineskip}%\noindent
}

\lstdefinestyle{mypython} {
  language=Python,
  otherkeywords={self},
  keywords=[100]{True,False,None},
  keywordstyle={[100]\color{builtincolor}},  
}
    
\lstnewenvironment{pycodeexample}[2][]{%
  \thicklines
  \lstset{
    style=mypython,
    backgroundcolor=\color{codeexbkg},
    %rulecolor=\color{green!40!black},
    frameround=tttt,
    frame=single,
    emph=[100]{#2},
    emphstyle={[100]\color{identifiercolor}},
    %basicstyle=\ttfamily\small,
    %basewidth=0.50em,
    %keywordstyle=\bfseries,
    #1%
  }
}{}



\title{SST Core Python Module Manual}
\date{\today}

\begin{document}
\maketitle
\newpage

\pagenumbering{Roman}

%\tableofcontents
%\newpage
%\listoffigures
%\newpage
%\listoftables
%\newpage
\pagenumbering{arabic}

\chapter{Introduction}

SST provides a python module to allow interaction with the simulation
build system.  This python module is used in the input python script
for the purpose of building the graph that represents the simulation
to be performed.  This is done by providing class and functions to
define the elements of the simluation, their parameters, and how they
are interconnected.  The user can also optionally enable statistics
and create a user specified partitioning for the described model.
Behind the scenes, these classes will build the c++ data structure
that is used by SST to construct the simulation model.

The SST core python module is defined in cpython and is only
available in the python interpreter launched within a running SST
executable.  The module is accessed by importing the sst module.  This
can be done in a number of ways.  The two most common being:

\begin{lstlisting}
# Import SST python module using sst. prefix
import sst
# Import SST python module members into current namespace
from sst import *
\end{lstlisting}

Within this module, there are a number of available classes and global
functions.  The available classes are: Component, SubComponent, Link,
StatisticOutput and StatisticGroup.  The global functions are divided
between general functions and functions operating on or returning one
of the available objects.  This document will first discuss the
available classes in the SST python module and will then document the
global functions.


\input{classes.tex}

\chapter{Global Functions in SST Python Module}
\label{chap:globals}

The SST core python module provides a set of global functions not
attached to any particular class. These functions generally fall into
one of the following categories: general control and informational
functions, functions to get handles to existing objects and statistic
enable and control functions.  These functions are described below.
Following those descriptions is a section on general notes on
statistics (Section~\ref{sec:gen-notes-stats}).

\section{General Control and Informational Functions}
\label{sec:sim-control}

\begin{functiondoc}{setProgramOption(option,value)}{
    Sets the specified program option for the simulation.  These
    mirror the options available on the sst command line.  Parameters
    set in the python file will be overwritten by options set on the
    command line.  Use sst --help to get a list of available options.
}

  \param{option} (type: string) configuration option to set

  \param{value} (type: varies by option) value to set option to

  \noreturn
\end{functiondoc}


\begin{functiondoc}{getProgramOptions()}{
    Returns a dictionary with the current values of the program
    options.  This will include all program options, not just those
    set in the python file.  }

  \returns python dictionary with program options and values
\end{functiondoc}


\begin{functiondoc}{getMPIRankCount()}{
    Returns the number of physical MPI ranks in the simulation
}
  
\returns number of MPI ranks in the simulation
\end{functiondoc}


\begin{functiondoc}{getSSTThreadCount()}{
    Returns the threads per rank specified for the simulation
}

  \returns number of threads per MPI rank in the simulation
\end{functiondoc}


\begin{functiondoc}{setSSTThreadCount(threads)}{
    Sets the number of threads per rank for the simulation.  These
    values can be overwritten by using -n on the command line.
}

  \param{threads} (type: int) number of threads per MPI rank to use in the
  simulation

  \noreturn
\end{functiondoc}

\begin{functiondoc}{pushNamePrefix(prefix)}{
    Pushes a name prefix onto the name stack.  This prefix will be
    added on the names of all Components and Links.  The names in the
    stack are separated by a period.  Example, if
    pushNamePrefix(“base”) and pushNamePrefix(“next”) were called in
    that order, the prefixed name would be “base.next”.  Prefixes can
    be popped from the stack using popNamePrefix().
}

  \param{prefix}: (type: string) prefix to add to the name stack

  \noreturn
\end{functiondoc}


\begin{functiondoc}{popNamePrefix()}{
    Pops a prefix from the name stack.  See pushNamePrefix for how
    name stacks are used.
  }

  \noreturn
\end{functiondoc}


\begin{functiondoc}{exit()}{
    Causes the simulation to exit.
}

  \noreturn
\end{functiondoc}
  

\section{Functions to Get Handles to Existing Objects}
\label{sec:handles}

\begin{functiondoc}{findComponentByName(name)}{
    In many cases, Components and SubComponents will be created using
    library functions and the user will not have direct access to
    their handles.  In some instances, the provided python modules
    will have accessor functions that can provide handles to these
    elements.  If this is not provided by the library, the user can
    call the findComponentByName() function to get a handle to the
    desired element.  The function can find handles for both
    Components and SubComponents.  The use of this function
    presupposes a knowledge of the naming convention of the elements
    in the build functions of the library.
}

  \param{name} (type: string) name of the Component or SubComponent to
  find.  The name for SubComponents is described above.  Slot indexes
  are optional in cases where only on SubComponent has been added to a
  slot, but you can also use [0] in all cases, even when the actual
  name will not display this way.

  \returns the function will return a handle to the
  Component/SubComponent with the provided name, or None if the name
  is not found.
\end{functiondoc}


\section{Statistic Enable and Control Functions}
\label{sec:stat-enable}

The following functions are used to enable statistics on Components
and SubComponents using the name or type of the element.  See “General
Notes on Statistics” (Section \ref{sec:gen-notes-stats}) below for more
information.

\begin{functiondoc}{enableAllStatisticsForAllComponents(stat_params_dict)}{
    Enables all statistics for all Components in the simulation that
    have already been instanced.
}
  
  \param{stat_params_dict} (type: dict) Python dictionary that specified the
  statistic parameters.  All statistics will get the same set of
  parameters.

  \noreturn
\end{functiondoc}


\begin{functiondoc}{enableAllStatisticsForComponentName(name, stat_params_dict,
    apply_to_children=False)}{

    Enables all statistics for the Component named in the call.  This
    call works for both Components and SubComponents.
}

  \param{name} (type: string) name of the Component or SubComponent on
  which to enable all statistics.  The name for SubComponents is
  described above.  Slot indexes are optional in cases where only one
  SubComponent has been added to a slot, but you can also use [0] in
  all cases, even when the actual name will not display this way.  If
  component with the provided name not found, the function will call
  fatal().

  \param{stat_params_dict} (type: dict) Python dictionary that
  specified the statistic parameters.  All statistics will get the
  same set of parameters

  \param{include_children} (type: bool) If set to True, will
  recursively enable all statistics on all SubComponent descendants of
  named element.

  \noreturn
\end{functiondoc}


\section{General Notes on Statistics}
\label{sec:gen-notes-stats}


\end{document}
